\documentclass{article}
    \linespread{1.75}
    \usepackage{amsmath}
    \usepackage{amssymb}
    \usepackage{listings}
    \title{\vspace{-2cm}Introduction to Computer Science - Homework 2}
    \author{Rumen Mitov}
\begin{document}
\maketitle

\section*{Problem 2.1}
\underline{Show:} $\forall a \in \mathbb{Z} \mid a^{32}$ is odd $\implies a^4$ is odd. \\
\underline{Proof by contrapositive:} \\
Consider that $a^4$ is even. \\
We know that $\forall n \in \mathbb{Z} \mid$ n is even $\implies n^2$ is even. \\
Hence: \\
${{{a^4}^2}^2}^2 = a^{32}$ is even. \\
$\therefore$ Contrapositive is true $\implies$ original statement is true. \\
$a^{32}$ is odd $\implies a^4$ is odd. $\blacksquare$ \\

\section*{Problem 2.2}
\underline{Show:} $\forall n \in \mathbb{N} \mid n \geq 1 \mid n^3 + (n + 1)^3 + (n + 2)^3$ is divisible by 9. \\
\underline{Proof by induction:} \\
\textbf{Base case:} \\
\[
    (1)^3 + (1 + 1)^3 + (1 + 2)^3  
\]
\[
    = 1 + 8 + 9
\]
\[
    = 18 \land 18 \equiv 0 \text{ (mod 9)}
\]
\textbf{Induction hypothesis:} \\
\[
    n^3 + (n + 1)^3 + (n + 2)^3
\]
\[
    = 3n^3 + 9n^2 + 15n + 9
\]
\textbf{Induction step:} $n \to n + 1$ \\
Substituting the \textit{induction step} into the \textit{induction hypothesis} results in: \\
\[
    ((n + 1) + 1)^3 + ((n + 1) + 2)^3 + ((n + 1) + 2)^3
\]
\[
    = (n + 1)^3 + (n + 2)^3 + (n + 3)^3
\]
\[
    = 3n^3 + 18n^2 + 42n + 36
\]
Now we check if the difference between the $n + 1$ -th term and the n-term is divisible by 9:
\[
    3n^3 + 18n^2 + 42n + 36
\]
\[
    - 3n^3 + 9n^2 + 15n + 9
\]
\[
    = 9n^2 + 27n + 27
\]
Notice that $\forall x \in \{9n^2, 27n, 27\} \mid x \equiv 0$ (mod 9). \\
$\therefore$ Since the difference between our \textit{induction hypothesis} and its subsequent term
are divisible by 9, then $\forall n \in \mathbb{N} \mid n \geq 1$ the \textit{induction 
hypothesis} is true. $\blacksquare$

\vspace{3cm}
\section*{Problem 2.3}
\begin{lstlisting}[language=Haskell, caption=Haskell code of divisor and sigma functions]
-- Return the list of positive divisors of an integer n.
divisors :: Int -> [Int]
divisors n = [ x | x <- [1..n], mod n x == 0 ]

-- Return the sum of divisors of n taken to the power of z
sigma :: Int -> Int -> Int
sigma z n = sum [x ^ z | x <- divisors n]
\end{lstlisting}
\end{document}
